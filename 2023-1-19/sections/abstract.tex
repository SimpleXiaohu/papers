%!TEX root = ../main.tex

%\documentclass{standalone}
%\begin{document}

Regular expressions (regex for short) and string-length function are widely used in string-manipulating programs. 
%Solving string constraints involving regular membership and string-length function has been widely investigated. 
Counting is a frequently used feature in regexes that counts the number of sub-patterns matchings. The state-of-the-art string solvers are incapable of solving string constraints with regex-counting and string-length efficiently, especially when the counting bounds are large. In this work, we propose an automata-theoretic approach for solving such a class of string constraints. 
%
The main idea is to model the counting operators symbolically by registers in automata instead of unfolding them explicitly, thus alleviating the state explosion problem.  
%
Moreover, the string-length function is modeled by register as well. 
%Since Parikh images of automata can be defined by linear integer arithmetic formulas, 
As a result, the satisfiability of string constraints with regex-counting and string-length is reduced to the satisfiability of linear integer arithmetic, which off-the-shelf SMT solvers can then solve. 
%
%we extract linear integer arithmetic formulas from automata, which encode the constraints on the values of registers, and utilize the off-the-shelf SMT solvers to solve the linear integer arithmetic constraints. 
%
To improve the performance further, we propose techniques to reduce automata sizes.
% we propose state reduction techniques and under-approximation to improve the performance further. 
We implement the algorithms and validate our approach on 48,843 benchmark instances. The experimental results show that 
our approach can solve more instances than other solvers at a comparable or faster speed.
% performance of this approach compared against the state-of-the-art string solvers.


%Regular expression with repetition frequently appears in the real world, such as $(ab)\{1, 100\}$. However, state-of-art string
%solvers can not efficiently solve string constraints with repetition bounds,
%especially when the repetition bounds are large. This paper focuses on
%string logic containing regular membership predicate with repetition
%and linear integer constraints on string length. Repetition is captured
%by cost-enriched finite automaton(CEFA), whose transitions have symbolic updates of integers limited by linear arithmetic. We propose a new systematic algorithm based on CEFA, and heuristic ways
%like under-approximation and symbolic-aware simplification are used to
%accelerate. We implement a powerful string solver OstrichCEA and
%evaluate it on instances from the real world and other typical benchmarks.
%The experiment result shows superiority.

%\keywords{Rgular Expressions, Counting, Length, Automata, Registers, Linear Integer Arithmetic}
%\end{document}