\section{Construction of Basic Regex Operations} \label{appendix:cefa}
In the sequel, we recall the constructions of NFA corresponding to \emph{union, intersection, concatenation, Kleene star and complement}. 
Let $\NFA_1 = (Q_1, \Sigma, \delta_1, I_1, F_1)$ and $\NFA_2 = (Q_2, \Sigma, \delta_2, I_2, F_2)$ be two NFA such that $Q_1 \cap Q_2 = \emptyset$. 
\begin{itemize}
\item The union of $\NFA_1$ and $\NFA_2$, denoted by $\NFA_1 \cup \NFA_2$, is defined as the tuple $(Q', \Sigma, \delta', I', F')$, where $Q'= Q_1 \cup Q_2$, $\delta' = \delta_1 \cup \delta_2$, $I' = I_1 \cup I_2$, and $F'=F_1 \cup F_2$. 
%
\item The intersection of $\NFA_1$ and $\NFA_2$, denoted by $\NFA_1 \cap \NFA_2$, is defined as the tuple $(Q', \Sigma, \delta', I', F')$, where $Q' = Q_1 \times Q_2$, $I' = I_1 \times I_2$, $F' = F_1 \times F_2$, and $\delta' = \{((q_1, q_2), a, (q'_1, q'_2)) \mid (q_1, a, q'_1) \in \delta_1, (q_2, a, q'_2) \in \delta_2\}$.
%
\item The concatenation of $\NFA_1$ and $\NFA_2$, denoted by $\NFA_1 \concat \NFA_2$, is defined as the tuple $(Q', \Sigma, \delta', I', F')$, where $Q' = Q_1 \cup Q_2$, $I' = I_1$, $\delta' = \delta_1 \cup \delta_2 \cup \{(q_1, a, q_2) \mid q_1 \in F_1, \exists q' \in I_2.\ (q', a, q_2) \in \delta_2\}$, moreover, if $I_2 \cap F_2 \neq \emptyset$, then $F'= F_1 \cup F_2$, otherwise, $F'= F_2$.
%
\item The Kleene star of $\NFA_1$, denoted by $\NFA_1^*$, is defined as $(Q'_1, \Sigma, \delta'_1, I'_1, F'_1)$, where $Q'_1 = Q_1 \cup \{q_0\}$ with $q_0 \not \in Q_1$, $I'_1 = \{q_0\}$, $F'_1= F_1 \cup \{q_0\}$, $\delta'_1 = \delta_1 \cup \{(q_0, a, q'_1)  \mid \exists q'_0 \in I_1.\ (q'_0, a, q'_1) \in \delta_1\} \cup \{(q_1, a, q'_1) \mid q_1 \in F_1, \exists q'_0 \in I_1.\ (q'_0, a, q'_1) \in \delta_1\}$.
%
\item The complement of $\NFA_1$, denoted by $\overline{\NFA_1}$, is defined as $(Q'_1, \Sigma, \delta'_1, q'_0, Q'_1 \setminus F'_1)$, where  $(Q'_1, \Sigma, \delta'_1, q'_0, F'_1)$ is the DFA obtained from $\NFA_1$ by subset construction.  
%
\item The language difference of $\NFA_1$ and $\NFA_2$, denoted by $\NFA_1 \setminus \NFA_2$, is defined as $\NFA_1 \cap \overline{\NFA_2}$.
\end{itemize}

% The detailed results for the \textbf{AutomatArk} benchmark are presented in Figure \ref{fig:cactus_automatrk} and Table \ref{tab:results_automatrk}. OstrichCEA solved the most \emph{unsat} instances. Z3str3RE solved the most \emph{sat} instances. Ostrich solved the greatest number of instances, while Z3str3RE used the least time with timeouts.\newline
% The detailed results for the \textbf{Redos} benchmark are presented in Figure \ref{fig:cactus_redos} and Table \ref{tab:results_redos}. OstrichCEA solved the most instances on both tracks of \emph{sat} and \emph{unsat}. Including timeouts, OstrichCEA is \textbf{4.4}\mult{} faster than CVC5, \textbf{3.74}\mult{} faster than Ostrich, \textbf{2.09}\mult{} faster than Z3str3, \textbf{5.40}\mult{} faster than Z3seq, \textbf{2.49}\mult{} faster than Z3-Trau and close to Z3str3RE with \%20 speed loss. \newline 
% The detailed results for the \textbf{RegexLib} benchmark are presented in Figure \ref{fig:cactus_regexlib} and Table \ref{tab:results_regexlib}. Ostrich solved the most \emph{unsat} instances. OstrichCEA solved the most \emph{sat} instances. In total, OstrichCEA solved the greatest number of instances and was the solver with medium speed. \newline
% The detailed results for the \textbf{StackOverflow} benchmark are presented in Figure \ref{fig:cactus_stackoverflow} and Table \ref{tab:results_stackoverflow}. OstrichCEA solved the most instances on both tracks of \emph{sat} and \emph{unsat} and was the faster solver.\newline Including timeouts, OstrichCEA is \textbf{3.98}\mult{} faster than CVC5, \textbf{2.63}\mult{} faster than Ostrich, \textbf{2.19}\mult{} faster than Z3str3, \textbf{1.59}\mult{} faster than Z3seq, \textbf{1.28}\mult{} faster than CVC5 and \textbf{1.47}\mult{} to Z3str3RE. \newline
% \begin{figure}
%   \centering
%   \subimport{figures}{cactus_plot_automatark.tex}
%   \caption{The plot of a cactus graph depicting a comprehensive evaluation of the performance of the AutomatArk benchmark.}
%   \label{fig:cactus_automatrk}
% \end{figure}
% \begin{table}
%   \subimport{tables}{table_automatark.tex}
%   \caption{Detailed results for the AutomatArk benchmark.}
%   \label{tab:results_automatrk}
% \end{table}

% \begin{figure}
%   \subimport{figures}{cactus_plot_redos.tex}
%   \caption{The plot of a cactus graph depicting a comprehensive evaluation of the performance of the ReDos benchmark.}
%   \label{fig:cactus_redos}
% \end{figure}
% \begin{table}
%   \subimport{tables}{table_redos.tex}
%   \caption{Detailed results for the ReDos benchmark.}
%   \label{tab:results_redos}
% \end{table}

% \begin{figure}
%   \subimport{figures}{cactus_plot_regexlib.tex}
%   \caption{The plot of a cactus graph depicting a comprehensive evaluation of the performance of the RegexLib benchmark.}
%   \label{fig:cactus_regexlib}
% \end{figure}
% \begin{table}
%   \subimport{tables}{table_regexlib.tex}
%   \caption{Detailed results for the RegexLib benchmark.}
%   \label{tab:results_regexlib}
% \end{table}

% \begin{figure}
%   \subimport{figures/}{cactus_plot_stackoverflow.tex}
%   \caption{The plot of a cactus graph depicting a comprehensive evaluation of the performance of the StackOverflow benchmark.}
%   \label{fig:cactus_stackoverflow}
% \end{figure}
% \begin{table}
%   \subimport{tables}{table_stackoverflow.tex}
%   \caption{Detailed results for the StackOverflow benchmark.}
%   \label{tab:results_stackoverflow}
% \end{table}
