%!TEX root = ../main.tex
%\documentclass{standalone}
%\begin{document}

In this section, we define the model of cost-enriched finite state automata (CEFA), which was introduced in \cite{atva2020} and will be used to solve the satisfiability problem of RECL later on. 
%
Intuitively, CEFA add\denghang{adds} write-only cost registers to finite state automata. Here by ``write-only'', we mean that the cost registers can only be written/updated, but cannot be read, that is, cannot be used to guard the transitions. 

%
%In~\cite{atva2020}, the cost function is defined as a function $\eta: \Sigma \rightarrow \mathbb{N}$. In this paper, we define the cost function as an integer vector whose elements are the incremental value of registers. Furthermore, we add a new linear integer arithmetic constraint $\theta$ to each final state of the CEFA, which restricts the value of registers. Two types of definitions have the same expressive ability on the $SAT_{CL}$ problem.
\begin{definition}[Cost-Enriched Finite Automaton]
  A cost-enriched finite automaton $\aut$ is a tuple $(R, Q, \Sigma, \delta, I, F)$ where
  \begin{itemize}
  \item $R = \{r_1, \cdots, r_k\}$ is a finite set of registers, 
    \item $Q, \Sigma, I, F$ are as in the definition of NFA,
%    \item $R = (r_1\cdots r_n)$ is a vector of mutually distinct cost registers,
    \item $\delta \subseteq Q \times \Sigma \times Q \times \Int^R$ is a transition relation, where $\Int^R$ denotes the updates on the values of registers.
%  set whose elements are tuples $(q, c, q', \myvec{v})$ where $q, q'$ are states of $Q$, $c$ is a letter in alphabet $\Sigma\cup\{\epsilon\}$ and $\myvec{v}$ is the cost update function for registers, which is an integer vector whose $i$-th element is the incremental value of register $r_i$.  $(q, a, q', \myvec{v})$ is written as $q\xrightarrow[\myvec{v}]{a} q'$ for readability.
    \item $\alpha \in \Phi(R)$ is an LIA formula specifying an accepting condition.
%    a linear integer arithmetic constraint function on final states. $\theta$ is called \emph{accepting condition}.
  \end{itemize}
\end{definition}
%The definition of CEFA above is slightly different from the definition in~\cite{atva2020} in the sense that accepting conditions on register values
%are attached to final states. 

For readability, from now on, we assume a linear order on $R$ and write $R$ as a vector $\myvec{r} = (r_1, \cdots, r_k)$. Accordingly, we write an element of $\Int^R$ as a vector $(v_1, \cdots, v_k)$, where for each $i \in [k]$, $v_i$ is the update on the value of $r_i$. We also write a transition $(q, a, q', \vec{v}) \in \delta$ as $q \xrightarrow[\vec{v}]{a} q'$.

The semantics of CEFA is define\denghang{defined} as follows. Let $\aut = (R, Q, \Sigma, \delta, I, F, \alpha)$ be a CEFA. 
A \emph{configuration} of $\aut$ is a pair $(q, \vec{v})$ where $q \in Q$ and $\vec{v}$ is a vector denoting the values of registers.  
%An \emph{initial configuration} of $\aut$ is $(q_0, \vec{0})$ with $q_0 \in I$, where the value of each register is zero. 
A \emph{run} of $\aut$ on a string $w = a_1 \cdots a_n$ is a sequence $q_0 \xrightarrow[\myvec{v_1}]{a_1} q_1 \cdots q_{n-1}\xrightarrow[\myvec{v_n}]{a_n} q_n$ such that $q_0 \in I$ and $q_{i-1} \xrightarrow[\myvec{v_i}]{a_i} q_i$ for each $i \in [n]$. A run $q_0 \xrightarrow[\myvec{v_1}]{a_1} q_1 \cdots q_{n-1}\xrightarrow[\myvec{v_n}]{a_n} q_n$ is \emph{accepting} if $q_n \in F$ and $\alpha(\myvec{v'}/R)$ is true, where $\myvec{v'} = \sum \limits_{j \in [n]} \myvec{v_j}$. The vector $\myvec{v'}= \sum \limits_{j \in [n]} \myvec{v_j}$ is called the \emph{cost} of an accepting run $q_0 \xrightarrow[\myvec{v_1}]{a_1} q_1 \cdots q_{n-1}\xrightarrow[\myvec{v_n}]{a_n} q_n$. Note here \denghang{that} we assume that the initial values of all registers are zero and $\sum \limits_{j \in [n]} \myvec{v_j}$ is the vector of register values after all the transitions in the run are executed. We use $\myvec{v'} \in \aut(w)$ to denote the fact that there is an accepting run of $\aut$ on $w$ whose cost is $\myvec{v'}$.  
We define the semantics of a CEFA $\aut$, denoted by $\Lang(\aut)$, as $\{(w; \myvec{v'}) \mid  \myvec{v'} \in \aut(w)\}$.  In particular, if $I \cap F \neq \emptyset$, then $(\varepsilon; \myvec{0}) \in \Lang(\aut)$. Moreover, we define the \emph{output} of a CEFA $\aut$, denoted by $\cefaout(\aut)$, as $\{\myvec{v'} \mid  \exists w.\ \myvec{v'} \in \aut(w)\}$.

We would like to remark that the accepting conditions $\alpha$ in CEFA are defined in a \emph{global} fashion in the sense that the accepting condition does not distinguish final states. This technical choice is made so that \denghang{the} determinization and minimization of NFA can be utilized to reduce the size of CEFA in the next section. 

%
%and $\alpha(q_n)[\myvec{\myvec{v'}/\myvec{r}]$ is true, where $\myvec{v'} = \sum \limits_{j \in [n]} \myvec{v_j}}$. Note here \denghang{that} we assume that the initial values of all registers are zero and $\sum \limits_{j \in [n]} \myvec{v_j}$ is the vector of register values after all the transitions in the run are executed. A string $w$ is \emph{accepted} by $\aut$ if there is an accepting run of $\aut$ on $w$. The language defined by $\aut$, denoted by $\Lang(\aut)$, is the set of strings that are accepted by $\aut$. 

%Finally, we define the product of two CEFA. 

For convenience, for a CEFA $\aut$, we use $R_\aut$ to denote the set of registers of $\aut$. 

In the sequel, we define three operations on CEFA, namely, union, intersection, and concatenation. These three operations will be used for solving RECL constraints in the next section. 

Let $\aut_1 = (R_1, Q_1, \Sigma, \delta_1, q_{1,0}, F_1, \alpha_1)$ and  $\aut_2 = (R_2, Q_2, \Sigma, \delta_2, q_{2,0}, F_2, \alpha_2)$ be two CEFA that share the alphabet. 
Moreover, suppose that $R_1 \cap R_2 = \emptyset$ and $Q_1 \cap Q_2 = \emptyset$. 
%We define two operations of CEFA, namely, the product and union, as follows. 

The \emph{union} of $\aut_1$ and $\aut_2$, denoted by $\aut_1 \cup \aut_2$, is defined in a slightly more involved manner than NFA, as a result of the technical choice that the accepting conditions of CEFA do not distinguish final states. Two fresh auxiliary registers, say $r'_1,r'_2 \not \in R_1 \cup R_2$, are introduced so that the accepting condition knows whether a run is from $\aut_1$ or from $\aut_2$. Specifically, $\aut_1 \cup \aut_2 = (R', Q', \Sigma, \delta', I', F', \alpha')$, where 
\begin{itemize}
\item $R' = R_1 \cup R_2 \cup \{r'_1, r'_2\}$, $Q' = Q_1 \cup Q_2 \cup \{q'_0\}$ with $q'_0 \not \in Q_1 \cup Q_2$, $I' = \{q'_0\}$, 
\item $\delta'$ is the union of $\{(q'_0, a, q'_1, (\myvec{v_1}, \myvec{0}, 1,0)) \mid \exists q_1 \in I_1.\ (q_1, a, q'_1, \myvec{v_1}) \in \delta_1\}$, $\{(q'_0, a, q'_2, (\myvec{0}, \myvec{v_2}, 0, 1)) \mid \exists q_2 \in I_2.\ (q_2, a, q'_2, \myvec{v_2}) \in \delta_2\}$, $\{(q_1, a, q'_1, (\myvec{v_1}, \myvec{0}, 0, 0)) \mid (q_1, a, q'_1, \myvec{v_1}) \in \delta_1\}$, and $\{(q_2, a, q'_2, (\myvec{0}, \myvec{v_2}, 0, 0)) \mid (q_2, a, q'_2, \myvec{v_2}) \in \delta_2\}$, where the last two components of the vectors denote the updates of $r'_1$ and $r'_2$, 

\item $F'$ and $\alpha'$ are defined as follows, 
\begin{itemize}
\item if $I_1 \cap F_1 \neq \emptyset$ or $I_2 \cap F_2 \neq \emptyset$, then $F' = F_1 \cup F_2 \cup \{q'_0\}$ and $\alpha' = (r'_1 = 0 \wedge r'_2 = 0) \vee (r'_1=1 \wedge \alpha_1) \vee (r'_2=1 \wedge \alpha_2)$, 
%
\item otherwise, $F'=F_1 \cup F_2$ and $\alpha' = (r'_1=1 \wedge \alpha_1) \vee (r'_2=1 \wedge \alpha_2)$.
\end{itemize}
\end{itemize}
From the construction, we know that 
$$\Lang(\aut_1 \cup \aut_2) = \{(w; \myvec{v}) \mid \myvec{v} \in \Int^{R'}, (w; \prj_{R_1}(\myvec{v})) \in \Lang(\aut_1) \mbox{ or } (w; \prj_{R_2}(\myvec{v})) \in \Lang(\aut_2)\}.$$ 

The \emph{intersection} of $\aut_1$ and $\aut_2$, denoted by $\aut_1 \cap \aut_2 = (R', Q', \Sigma, \delta', I', F', \alpha')$, is defined in the sequel. 
\begin{itemize}
\item $R' = R_1 \cup R_2$, $Q' = Q_1 \times Q_2$, $I' = I_1 \times I_2$, $F' = F_1 \times F_2$, $\alpha' = \alpha_1 \wedge \alpha_2$, 
%
%\item $\alpha'((q_1, q_2)) = \alpha_1(q_1) \wedge \alpha_2(q_2)$ for each $(q_1, q_2) \in F'$, 
%
\item $\delta'$ comprises the tuples $((q_1, q_2), a, (q'_1, q'_2), (\myvec{v_1}, \myvec{v_2}))$ such that $(q_1,a, q'_1, \myvec{v_1}) \in \delta_1$ and $(q_2,a, q'_2, \myvec{v_2}) \in \delta_2$.
%, where $\myvec{v} \in \Int^{R'}$, $\prj_{R_1}(\myvec{v}))$ (resp. $\prj_{R_2}(\myvec{v}))$) is the restriction of $\myvec{v}$ to the domain $R_1$ (resp. $R_2$). 
\end{itemize}
From the construction, 
$$\Lang(\aut_1 \cap \aut_2) = \{(w; \myvec{v}) \mid \myvec{v} \in \Int^{R'}, (w; \prj_{R_1}(\myvec{v})) \in \Lang(\aut_1) \mbox{ and } (w; \prj_{R_2}(\myvec{v})) \in \Lang(\aut_2)\}.$$ 
Intuitively, $\aut_1 \cap \aut_2$ accepts the words that are accepted by both $\aut_1$ and $\aut_2$ and outputs the costs of $\aut_1$ and $\aut_2$.  

The \emph{concatenation} of $\aut_1$ and $\aut_2$, denoted by $\aut_1 \concat \aut_2$, is defined similarly as that of NFA, that is, a tuple $(Q', \Sigma, \delta', I', F', \alpha')$, where $Q' = Q_1 \cup Q_2$, $I' = I_1$, $\alpha' = \alpha_1 \wedge \alpha_2$, $\delta' = \{(q_1, a, q'_1, (\myvec{v_1}, \myvec{0})) \mid (q_1, a, q'_1, \myvec{v_1}) \in \delta_1\} \cup \{(q_2, a, q'_2, (\myvec{0}, \myvec{v_2})) \mid (q_2, a, q'_2, \myvec{v_2}) \in \delta_2\} \cup \{(q_1, a, q_2, \vec{v_2}) \mid q_1 \in F_1, \exists q' \in I_2.\ (q', a, q_2, \vec{v_2}) \in \delta_2\}$, moreover, if $I_2 \cap F_2 \neq \emptyset$, then $F'= F_1 \cup F_2$, otherwise, $F'= F_2$. From the construction, 
$$\Lang(\aut_1 \concat \aut_2) = \{(w_1w_2; \myvec{v}) \mid (w_1; \prj_{R_1}(\myvec{v})) \in \Lang(\aut_1), (w_2; \prj_{R_2}(\myvec{v})) \in \Lang(\aut_2)\}.$$

Furthermore, the union, intersection, and concatenation operations can be extended naturally to multiple CEFA, that is, $\aut_1 \cup \cdots \cup \aut_n$, $\aut_1 \cap \cdots \cap \aut_n$, $\aut_1 \concat \cdots \concat \aut_n$. For instance, $\aut_1 \cup \aut_2 \cup \aut_3 = (\aut_1 \cup \aut_2) \cup \aut_3$, $\aut_1 \cap \aut_2 \cap \aut_3 = (\aut_1 \cap \aut_2) \cap \aut_3$, and $\aut_1 \concat \aut_2 \concat \aut_3 = (\aut_1 \concat \aut_2) \concat \aut_3$.

%Let $((\aut_1,\cdots, \aut_n), \varphi)$ be a tuple such that $\aut_1 = (R_1, Q_1, \Sigma, \delta_1, q_{1,0}, F_1)$, $\cdots$, $\aut_n = (R_n, Q_n, \Sigma, \delta_n, q_{n, 0}, F_n)$ are CEFA, and $\varphi(R_1, \cdots, R_n)$ is an LIA formula. Then the \emph{relation} defined by $((\aut_1, \cdots, \aut_n), \varphi)$, denoted by $\srel((\aut_1, \cdots, \aut_n), \varphi)$, is $\{(w, \myvec{v'_1}, \cdots, \myvec{v'_n}) \mid \varphi(\myvec{v'_1}/R_1, \cdots, \myvec{v'_n}/R_n)\equiv \ltrue, \forall i \in [n].\ (w, \myvec{v'_i}) \in \Lang(\aut_i)\}$.

%the intersection of $\Lang(\aut_1)$ and $\Lang(\aut_2)$.

%The value of each register $r_i$ in $R$ is written as $\mathcal{V}(r_i)$ and is initiated to 0 at the initial state. A \emph{run} of $\aut$ on string $a_1\cdots a_m$ is a transition sequence $q_I\xrightarrow[\myvec{v_1}]{a_1}q_1\cdots q_{m-1}\xrightarrow[\myvec{v_m}]{a_m}q_m$  and $\mathcal{V}(r_i) = \displaystyle\sum_{j=1}^m \myvec{v_j}[i], i\in [1,n]$ is the value of $r_i$ after the run. $\theta[R/\mathcal{V}(R)](q_m)$ is obtained from $\theta(q_m)$ by replacing each register $r_i$ to its value $\mathcal{V}(r_i)$. The run is \emph{accepting} if $q_m\in F$ and $\theta[R/\mathcal{V}(R)](q_m)$ is satisfiable. $\top$ is the valid formula that is always satisfiable. A string $w$ is accepted by $\aut$ if it has an accepting run of $\aut$. The language of $\aut$ is the set of strings accepted by $\aut$, denoted by $\lan(\aut)$.

%%%%%%%%%%%%%%%%%%%%%%%%%%%%%%%%%%%%%%
%%%%%%%%%%%%%%%%%%%%%%%%%%%%%%%%%%%%%%
\hide{
\zhilin{this example should be modified to match the running example}
\begin{example} \label{eg:1}
  Fig.\ref{fig:counting} illustrates the CEFA recognizing $(ab)\{1,100\}$. The register vector is $(r_1)$, and the linear arithmetic constraint on $q_2$ is $1\leq r_1\leq 100$. Intuitively, the transition $q_0 \xrightarrow[(1)]{a} q_1$ accept the char $a$ and increase 1 on the value of $r_1$. The transition $q_1 \xrightarrow[(0)]{b} q_2$ accepts the char $b$ and does not change the value. The transition $q_2 \xrightarrow[(0)]{\epsilon} q_2$ is a nondeterministic choice back to initial state $q_0$. Because of the linear arithmetic constraint $1\leq r_1\leq 100$, the value of $r_1$ in the accepting run must be $[1,100]$. $\mathcal{V}(r_1)$ equal to the occurrence number of the transition $q_0 \xrightarrow[(1)]{a} q_1$, so that the accepted string is $ab\cdots ab$ in which $a$ appears at least once and at most 100 times. That is, the language of the automaton is $(ab)\{1,100\}$.
  \begin{figure}[h]
    \centering
    
\begin{tikzpicture}[
  shorten >=1pt,node distance=3cm,on grid,>={Stealth[round]},
  initial text=, every state/.style={minimum size = 0.001cm},
  accepting text=$\scriptstyle5\leq r_1\leq 60\wedge 5\leq r_2\leq 60\wedge 0\leq r_3\leq 60$, accepting/.style=accepting by arrow,
  accepting where=above
  ]

  \node[state,initial]            (q_0)                      {$q_0$};
  \node[state]                    (q_1) [right=of q_0]       {$q_1$};
  \node[state, accepting]                    (q_2) [right=of q_1]       {$q_2$};

  \path[->] (q_0) edge node [above] {$\scriptstyle \Sigma \setminus a:(1,0,0)$} (q_1)
            (q_0) edge [loop below] node  {$\scriptstyle\Sigma \setminus a:(1,0,0)$} (q_0)
            (q_1) edge node [above] {$\scriptstyle\Sigma \setminus b:(0,1,0)$} (q_2)
            (q_1) edge [loop below] node  {$\scriptstyle\Sigma \setminus b:(0,1,0)$} (q_1)
            (q_2) edge [loop below] node  {$\scriptstyle\Sigma \setminus c:(0,0,1)$} (q_2);
\end{tikzpicture}

% \begin{tikzpicture}[
%   shorten >=1pt,node distance=2cm,on grid,>={Stealth[round]},
%   initial text=, every state/.style={minimum size = 0.001cm,
%       accepting text=$\scriptstyle 1\leq r_1\leq 100$, accepting/.style=accepting by arrow}
%   ]
%   \node[state,initial]            (q_0)                      {$q_0$};
%   \node[state]                    (q_1) [right=of q_0]       {$q_1$};
%   \node[state,accepting]          (q_2) [right=of q_1]       {$q_2$};

%   \path[->] (q_0) edge              node      [above]           {$\scriptstyle a:(1)$} (q_1)
%   (q_1) edge              node      [above]           {$\scriptstyle b:(0)$} (q_2)
%   (q_2) edge [bend left]  node      [below]           {$\scriptstyle\epsilon: (0)$} (q_0);
% \end{tikzpicture}


    \caption{An automaton recognizing the language of $(\Sigma \setminus a)^{\{5, 60\}} (\Sigma \setminus b)^{\{5, 60\}} (\Sigma \setminus c)^{\{0, 60\}}$}
    \label{fig:counting}
  \end{figure}
\end{example}
}
%%%%%%%%%%%%%%%%%%%%%%%%%%%%%%%%%%%%%%
%%%%%%%%%%%%%%%%%%%%%%%%%%%%%%%%%%%%%%

% Parikh's theorem \cite{parikh_theorem}\cite{parikh_anthony}\cite{parikh_compute} states that the Parikh image of a context-free language is semilinear so that it can be written as Presburger formula. Following the version for context-free grammar \cite{parikh_compute} and the version for regular expression \cite{parikh_for_nfa}, we define the Parikh image of language of an CEFA $\aut = (Q, \Sigma, \delta, q_I, F, R, \theta)$  as follows:
% \begin{table}[h]
%   \begin{tabular}{l l r}
%     $\psi(\aut)\equiv$& $\bigwedge\limits_{q\in Q}(\text{$f_q$ if $q\in F$ otherwise 0}) + \sum\limits_{(q,a,q',\myvec{v})\in\delta} t_{q,a,q',\myvec{v}} = $ & \\
%     & \quad \ \ (1 if $q = q_I$ otherwise 0) + $\sum\limits_{(q',a,q,\myvec{v})\in\delta} t_{q',a,q, \myvec{v}} $ & \\ 
%     & $\bigwedge\limits_{(q,a,q',\myvec{v})\in\delta} t_{q,a,q',\myvec{v}}\geq 0$ & (consistent formula)\\
%     & & \\
%     & $\bigwedge\limits_{(q,a,q',\myvec{v})\in\delta} t_{q,a,q',\myvec{v}}>0\rightarrow z_{q'} > 0$ & \\
%     & $\bigwedge\limits_{q\in F} z_q > 0\rightarrow f_q = 1$ & \\ 
%     & $\bigwedge\limits_{q\in Q} z_q > 0\rightarrow \bigvee\limits_{(q,a,q',\myvec{v})\in\delta} z_q = z_{q'} + 1\wedge t_{q,a,q',\myvec{v}}\geq 0 \wedge z_{q'} > 0$& (connected formula)\\
%     & & \\
%     &$\bigwedge\limits_{r_i\in R} r_i = \sum\limits_{(q,a,q',\myvec{v})\in \delta} t_{q,a,q',\myvec{v}}*\myvec{v}[i]$ & (register value formula)\\
%     & & \\
%     &$\bigwedge\limits_{q\in F} f_q > 0 \rightarrow \theta(q)$ & (accepting condition) 
%   \end{tabular}
% \end{table}
% $z_q$ is the distance of state $q$ from $q_f\in F$ in a spanning tree. $t_{q,a,q',\myvec{v}}$ represents the using times of transition $(q,a,q',\myvec{v})$ in the accepting run. $f_q$ stands for whether an accepting state $q\in F$ is the final state of a run. The Parikh image of $\aut$ is a Presburger formula $\psi(\aut)$, which is consistent and connected. The register value formula ensures \denghang{that} the register values are correctly updated regarding the accepting run. 
% \begin{example} \label{exapmle:parikh}
%   Suppose that we have a CEFA $\aut$ shown in Fig. \ref{subfig:aut_x}. Using $t_1, t_2, t_3$ to record the times of transitions in the accepting run of $\aut$, the Parikh image $\psi(\aut)$ is the quantifier-free Presburger formula $\psi_{cons}\wedge \psi_{conn} \wedge \psi_{val} \wedge \psi_{aut}$ where:
%   \begin{itemize}
%     \item $\psi_{cons} \equiv 1 = t_1 \wedge t_1 + t_3 = t_2 \wedge t_2 = t_3 + 1 \bigwedge\limits_{i\in[1,3]} t_i\geq 0$ is the consistent formula.
%     \item $\psi_{conn} \equiv \top $ is the connectivity formula because all transitions in $\aut$ are connected.
%     \item $\psi_{val} \equiv r_1=t_1 + t_3\wedge r_2 = t_1+t_2+t_3$ is the formula deciding the value of $r_1$ and $r_2$ after a run.
%     \item $\psi_{aut} \equiv r_2 = i\wedge 1 \leq r_1 \leq 100$ is the accepting condition of $\aut$.
%   \end{itemize}
%   $\psi(\aut)$ can be simplified to $t_3\geq 0 \wedge t_1=1\wedge r_1 = t_3 + 1\wedge r_2 = 2*t_3+2\wedge r_2=i\wedge 1\leq r_1\leq 100$. A possible solution is $t_1 =1, t_3 = 0, r_1=1, r_2=2$ and the corresponding accepted string $ab$. It is obvious that $r_1$ is the repeat times of $ab$ and $r_2$ is the length of $ab$.
% \end{example}
%\end{document}