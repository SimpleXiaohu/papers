%!TEX root = ../main.tex
%\documentclass{standalone}
%\begin{document}

We write $\Nat$ and $\Int$ for the sets of natural and integer numbers, respectively. For $n \in \Nat$ with $n \ge 1$, $[n]$ denotes $\{1, \ldots, n\}$; for $m,n \in \Nat$ with $m \le n$,  $[m, n]$ denotes $\{ i \in \Nat \mid m \le i \le n \}$. Throughout the paper, $\Sigma$ is a finite alphabet, ranged over by $a,b,\ldots$.  

\paragraph*{Strings and languages.}
A string over $\Sigma$ is a (possibly empty) sequence of elements from $\Sigma$,
denoted by $u, v, w, \ldots$. An empty string is denoted by $\varepsilon$.  We write $\Sigma^*$ (resp., $\Sigma^+$) for the set of all (resp. nonempty) strings over $\Sigma$.
%Let 
For a string $u$, we use $|u|$ to denote the number of letters in $u$. In particular, $|\varepsilon|=0$. Moreover, for $a \in \Sigma$, let $|u|_a$ denote the number of occurrences of $a$ in $u$. 
Assume $u=a_0\cdots a_{n-1}$ is nonempty and $i<j \in [0,n-1]$. %Then a \emph{position} of $u$ is a number $i \in [|u|]$ (Note that the first position is $1$, instead of  0). In addition, 
We let $u[i]$ denote $a_i$ and $u[i,j]$ for the substring %of $u$ starting from $i$ and ending with $j$ (i.e., 
$a_i\cdots a_j$. 
%\tl{check later for consistency}\zhilin{the indices start from 0, to be consistent with the semantics of substring and indexof}\zhilin{i do use it in section 4}

Let $u, v$ be two strings. We use $u \cdot v$ to denote the \emph{concatenation} of $u$ and $v$.
%, that is, the string $w$ such that $|w|= |u| + |v|$ and for each $i \in [0, |u|-1]$, $w[i]= u[i]$, and for each $i \in [0,|v|-1]$, $w[|u|+i]=v[i]$. 

\paragraph*{Finite state automata.} 
A \emph{(nondeterministic) finite state automaton} (NFA)  is a tuple $\NFA=(Q, \Sigma, \delta, I, F)$, where $Q$ is a finite set of states, $\Sigma$ is a finite alphabet, $\delta \subseteq Q \times \Sigma \times Q$ is the transition relation, $I,F \subseteq Q$ are the set of initial and final states respectively. For readability, we write a transition $(q, a, q') \in \delta$ as $q \xrightarrow[\delta]{a} q'$ (or simply $q \xrightarrow{a} q'$). %Moreover, when $\delta$ is clear from context, we omit $\delta$ in $q \xrightarrow[\delta]{a} q'$ and write $q \xrightarrow{a} q'$. 
The \emph{size} of an NFA $\NFA$, denoted by $|\NFA|$, is defined as the number of transitions of $\NFA$.
%
A \emph{run} of $\NFA$ on a string $w = a_1 \cdots a_n$ is a sequence of transitions $q_0 \xrightarrow{a_1} q_1 \cdots q_{n-1} \xrightarrow{a_n} q_n$ with $q_0 \in I$. The run is \emph{accepting} if $q_n \in F$.
A string $w$ is accepted by an NFA $\NFA$ if there is an accepting run of $\NFA$ on $w$. In particular, the empty string $\varepsilon$ is accepted by $\NFA$ if $I \cap F \neq \emptyset$. The language of $\NFA$, denoted by $\Lang(\NFA)$, is the set of strings accepted by $\NFA$. 
%
An NFA $\NFA$ is said to be \emph{deterministic} if $I$ is a singleton and, for every $q \in Q$ and $a \in \Sigma$, there is at most one state $q' \in Q$ such that $(q, a, q') \in \delta$.
%
It is well-known that finite automata capture regular languages precisely. 

%\tl{to put in a different place later} Note that our decision procedure is applicable to general transducers as well, however, the EXPSPACE complexity bound is not, because the distributive property $f^{-1}(L_1\cap L_2)= f^{-1}(L_1)\cap f^{-1}(L_2)$ for regular languages $L_1, L_2$ only holds for functional transducers $f$.  

%We remark that an FT usually defines a relation.

%\smallskip

%In this paper, we consider logics involving two data-types, i.e., the string data-type and the integer data-type. We will use $u, v, \dots$ to denote string constants,  $c, d,\dots$ to denote integer constants, $x, y, \dots$ to denote string variables, and $i, j, \dots$ to denote  integer variables.

We will also use standard  quantifier-free/existential \emph{linear integer arithmetic} (LIA) formulae, which are typically ranged over by $\phi, \varphi$, etc. %, which is essentially Presburger 


%%%%%%%%%%%%%%%%%%%%%%%%%%%%%%%%%%
%%%%%%%%%%%%%%%%%%%%%%%%%%%%%%%%%%
\hide{
A finite \emph{alphabet} $\Sigma$ is the set of all \emph{letters}. A
\emph{string} (or \emph{word}) is a finite sequence of letters from $\Sigma$. $\Sigma^*$ is the set of strings over $\Sigma$. $\epsilon$ is the empty string. $L$ is grammar. A language $\lan(L)$ is a set of words generated by $L$. $\mathbb{N}$ is the set of natural numbers and $\mathbb{Z}$ is the set of integer numbers. We use $a,
  b,\cdots$ to denote the constant letters in $\Sigma$, $u, v,\cdots$ to denote constant
string, $x, y,\cdots$ to denote variable string, $m,n,\cdots$ to
denote integer constant, and $i,j\cdots$ to denote integer variable. For vector, we use $\myvec{v}$ to denote the vector of integer constant, $m_n$ to denote the vector $(m,\cdots, m)$ with length $n$, $\myvec{v}[i]$ to denote the integer value of $\myvec{v}$ at position $i$, $\myvec{v_1}\cdot\myvec{v_2}$ to denote the concatenation of $\myvec{v_1}$ and $\myvec{v_2}$, $R$ to denote the vector of registers, $R_1 \cap R_2$ to denote the same registers in $R_1$ and $R_2$, and $|\myvec{v}|$ or $|R|$ to denote the length of the vector.
}
