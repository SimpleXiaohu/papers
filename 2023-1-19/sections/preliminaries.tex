%!TEX root = ../main.tex
%\documentclass{standalone}
%\begin{document}

% Tokens
%\subsubsection{Tokens}

A finite \emph{alphabet} $\Sigma$ is the set of all \emph{letters}. A
\emph{string} (or \emph{word}) is a finite sequence of letters from $\Sigma$. $\Sigma^*$ is the set of strings over $\Sigma$. $\epsilon$ is the empty string. $L$ is grammar. A language $\lan(L)$ is a set of words generated by $L$. $\mathbb{N}$ is the set of natural numbers and $\mathbb{Z}$ is the set of integer numbers. We use $a,
  b,\cdots$ to denote the constant letters in $\Sigma$, $u, v,\cdots$ to denote constant
string, $x, y,\cdots$ to denote variable string, $m,n,\cdots$ to
denote integer constant, and $i,j\cdots$ to denote integer variable. For vector, we use $\myvec{v}$ to denote the vector of integer constant, $m_n$ to denote the vector $(m,\cdots, m)$ with length $n$, $\myvec{v}[i]$ to denote the integer value of $\myvec{v}$ at position $i$, $\myvec{v_1}\cdot\myvec{v_2}$ to denote the concatenation of $\myvec{v_1}$ and $\myvec{v_2}$, $R$ to denote the vector of registers, $R_1 \cap R_2$ to denote the same registers in $R_1$ and $R_2$, and $|\myvec{v}|$ or $|R|$ to denote the length of the vector.

Finite state automata
