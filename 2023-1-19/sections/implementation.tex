%!TEX root = ../main.tex
%\documentclass{standalone}
%\begin{document}

We implement the algorithms in Section~\ref{sec:algorithm} on top of OSTRICH~\cite{ostrich,atva2020}, resulting into\denghang{in} a version of OSTRICH called OSTRICH$^{\rm RECL}$.
Then we do experiments on two benchmark suites, to evaluate the performance of OSTRICH$^{\rm RECL}$. In the sequel, we first describe the two benchmark suites, then present the settings and results of the experiments. 

%This section presents the empirical evaluation of OstrichCEA, which is our implementation of the decision procedure introduced in Section \ref{sec:algorithm}. Our objective is to validate the effectiveness of the proposed techniques by evaluating our tool's correctness and efficiency compared to other solvers. Furthermore, we assess the efficacy of our heuristics by testing different configurations of the tool. We have implemented our encoding for counting with two heuristic algorithms on Ostrich+ \cite{atva2020}. The pre-image computation for concatenation, \verb|indexOf|, \verb|substring|, \verb|replaceAll|, \verb|reverse| and finite transducer remain unchanged. OstrichCEA is written in Scala and based on the SMT solver Princess\cite{princess}.

\subsection{Benchmark suites}

Our experiments utilize two benchmark suites, namely, AutomatArk and RegCoL. In total, there are 49,379 instances in the two benchmark suites. All the benchmark instances are in SMTLIB2 format. 

\paragraph*{AutomatArk benchmark suite.}
The AutomatArk benchmark suite is adapted from the AutomatArk suite in \cite{z3str3re} by filtering out the string constraints that contain no occurrences of counting operators and adding the length constraint $|x| > 10$ (where $x$ is a string variable). There are 8,751 instances in the AutomatArk suite here.
Note that the original AutomatArk benchmark suite in \cite{z3str3re} includes 19,979 instances in total, which are conjunctions of regular membership queries and generated out of regular expressions from \cite{automatark}. 
%Only 8,751 ones that contain counting operators out of 19,979 instances. 

\paragraph*{RegCoL benchmark suite.} There are 40,628 RECL constraints in the RegCoL suite. These constraints are generated by extracting regular expressions with counting operators from \cite{regex_lingua_franca,redos_lenka} and constructing for each regular expression $e$ a RECL constraint of the form 
$$x \in e \wedge x \in \overline{\Sigma^*(<+ >+'+''+\&)\Sigma^*} \wedge |x| > 10,$$   
where $\Sigma^*(<+ >+'+''+\&)\Sigma^*$ is a regular expression that specifies at least one occurrence of special characters $<$, $>$, $'$, $''$, or $\&$, and $\overline{e'}$ denotes the complement of a regular expression $e'$. We would like to remark that there are about 500,000 real-world regular expressions in \cite{regex_lingua_franca,redos_lenka} and regular expressions with counting operators occupy about 8\% of them. 

%%%%%%%%%%%%%%%%%% the benchmark suite written by Denghang %%%%%%%%%%%
%%%%%%%%%%%%%%%%%% the benchmark suite written by Denghang %%%%%%%%%%%
\hide{
We conducted a comparison on four sets of benchmarks based on regex with 
counting operator, consisting of a total of 49,379 instances. We analyze all 19 developed benchmarks listed in \cite{zaligvinder_2021} and find that only \textbf{AutomatArk} benchmarks have regular membership with counting operator. In total, almost 18\% of the instances we evaluated were sourced from published industrial benchmarks or other solver developers. The other instances are generated by ourselves with regular expressions from real world. Each set of the benchmark is evenly divided into "large" and "small": the "large" set contains instances with large upper bounds of counting operator (the sum of upper bounds is greater than 50), and "small" contains remained instances. About 10\% of the benchmarks are in a "large" set. More details about the benchmarks are shown below.

\subsubsection{AutomatArk} is the 8,751 instances generated by Berzish et al.\cite{z3str3re}. It is based on real-world regular expression queries from Loris D'Antoni\cite{automatark}. The origin set comprises two tracks, a simple and a hard track, with 19,979 instances. The simple track contains instances with a single regular expression membership constraint, whereas the hard track can hold up to five membership constraints for a single variable per instance. We extract 8,751 instances containing counting operator.

\subsubsection{ReDos} is the set of 1,624 instances we generated. It is based on the ReDos-attacked regular expression collected by Lenka et al. For each regular expression, we generate an instance as the template (\ref{eq:template}) where $\regex$ is the regular expression. The regular membership predicate $x\not\in \Sigma^*(<\mid >\mid '\mid ''\mid \&)\Sigma^*$ sanitizes the input string $x$ to avoid the attack. The length lower bound is set to $20$ for the "small" set and $50$ for the "large" set.

\subsubsection{RegexLib} is the set of 1,623 instances we generated similarly to the \textbf{ReDos} benchmark. It is based on the regular expressions collected by James C. Davis et al.\cite{regex_lingua_franca} from regex lib website\cite{regexlib}. The website is the Internet's first Regular Expression Library. Currently, it has indexed 4149 expressions from 2818 contributors around the world since 2001. We extract 1,623 instances containing bounded repetition from 4149 instances. The length lower bound is set to $20$ for the "small" set and $50$ for the "large" set.

\subsubsection{StackOverflow} is the 37381 instances we generated similarly to the \textbf{ReDos} benchmark. As the Regexlib benchmark, the real-world regex expressions are collected by James C. Davis et al.\cite{regex_lingua_franca} from StackOverflow website\cite{stackoverflow}. The website is a question-and-answer site for professional and enthusiast programmers. We extract 37,381 instances containing bounded repetition from almost 500,000 instances. The length lower bound is set to $20$ for the "small" set and $50$ for the "large" set.
\begin{equation} \label{eq:template}
  x\in \regex \wedge x\not\in \Sigma^*(<\mid >\mid '\mid ''\mid \&)\Sigma^*\wedge |x| > 50(or \ 20)
\end{equation}
}
%%%%%%%%%%%%%%%%%% the benchmark suite written by Denghang %%%%%%%%%%%
%%%%%%%%%%%%%%%%%% the benchmark suite written by Denghang %%%%%%%%%%%

\subsection{Experiment results}

We evaluated the performance of OSTRICH$^{\rm RECL}$ against the state-of-the-art string constraint solvers, including CVC5 
%(version 1.0.5) 
\cite{cvc5}, Z3seq \cite{z3seq}, Z3-Trau 
%(github commit 1628747) 
\cite{z3trau}, Z3str3
%(github commit 59e9c87) 
\cite{z3str3}, Z3str3RE \cite{z3str3re}, and OSTRICH\footnote{The name OSTRICH+ was used in~\cite{atva2020}, where nuXmv model checker was called to solve the nonemptiness of CEFA w.r.t. LIA. } 
%(github commit 8297d8d) 
\cite{atva2020}. All experiments are conducted on CentOS Stream release 8 with 12 Intel(R) Xeon(R) Platinum 8269CY 3.10GHz CPU cores and 190 GB memory. We use the Zaligvinder framework \cite{zaligvinder_2021} to execute the experiments, where the timeout period is set as 60 seconds per instance. 

The experiments are designed to answer the following two research questions. 
\begin{description}
\item[Q1] Does OSTRICH$^{\rm RECL}$ solve RECL constraints more efficiently than the state-of-the-art string constraint solvers ?

\item[Q2] Do the size-reduction and under-approximation techniques proposed in Section~\ref{sec:algorithm} indeed improve the performance ?
\end{description}

Q1 is answered by running OSTRICH$^{\rm RECL}$ and the aforementioned solvers on AutomatArk and RegCoL. The experimental results can be found in Table~\ref{tab:results_all}. From Table~\ref{tab:results_all}, we can see that OSTRICH$^{\rm RECL}$ solves 49,483 instances correctly, which is 4,253/2,561/27,184/13,787/1,407/\zhilin{xxx} more than the number of instances solved by CVC5/Z3seq/Z3-Trau/Z3str3/Z3str3RE/OSTRICH respectively. 
%    
Note that Z3seq, Z3-Trau, Z3str3, and Z3str3RE have soundness errors, where the result of CVC5 is taken as the ground truth. If CVC5 reports ``unknown'' or ``timeout'' in some instance, then no soundness errors will be reported in this instance. 
%A soundness error is reported if the result of a solver for an instance is different from CVC5. 
% their results on some instances are inconsistent with the results of the other three solvers, i.e. CVC5, OSTRICH, and OSTRICH$^{\rm RECL}$.
%
Moreover, OSTRICH$^{\rm RECL}$ is almost as fast as the fastest solver Z3str3RE. More specifically, OSTRICH$^{\rm RECL}$ spends 1.77 seconds per instance on average, while Z3str3RE spends 1.61 seconds per instance, while the other solvers spend at least 3 seconds per instance.  
Z3-Trau reports ``unknown'' on 21,152 instances, which is mainly attributed to the fact that it does not support \verb|re.diff|, the language difference operator of regular expressions. 

%\zhilin{may add the results for the two benchmark suites separately.}

\begin{table}[ht]
\begin{center}
  \import{tables}{table_all.tex}
\end{center}
  \caption{Experiment results for Q1, where the timeout period is 60 seconds}
  \label{tab:results_all}
\end{table}


To answer Q2, we compare OSTRICH$^{\rm RECL}$ with its three variants, namely, OSTRICH$^{\rm RECL}_{\rm -SR}$, OSTRICH$^{\rm RECL}_{\rm -UA}$, OSTRICH$^{\rm RECL}_{\rm -SRUA}$, which remove size-reduction, under-approximation, and both from OSTRICH$^{\rm RECL}$ respectively. The experiment results can be found in Table~\ref{tab:results_heuristics}.
From Table~\ref{tab:results_heuristics}, we can see that adding size-reduction techniques help solve \zhilin{xxx} more instances and reduce the average time per instance for \zhilin{xxx}/\%, while adding under-approximation techniques help solve \zhilin{xxx} more instances and reduce the average time per instance for \zhilin{xxx}/\%, while adding both help solve \zhilin{xxx} more instances and reduce the average time per instance for \zhilin{xxx}/\%. 
%Moreover, adding these techniques also decrease the average time spent in solving each instance considerably.

\begin{table}[ht]
\begin{center}
  \subimport{tables}{table_heuristic.tex}
\end{center}
  \caption{Experiment results for Q2, where the timeout period is 60 seconds}
  \label{tab:results_heuristics}
\end{table}

\zhilin{stopped here}

%%%%%%%%%% original texts by denghang %%%%%%%%%
%%%%%%%%%% original texts by denghang %%%%%%%%%
\hide{
%
We have evaluated OstrichCEA compared to five other prominent string solvers currently available. We evaluate the solvers by directly comparing the number of cases correctly solved, the average time taken with and without timeouts, and the total count of soundness errors and program crashes. One of these solvers is CVC5\cite{cvc5}, a general SMT solver that uses algebraic reasoning to handle strings and regular expressions and is the winner of SMT-COMP 2022\cite{smt-comp}. Another solver, Z3str3\cite{z3str3}, is the most recent addition we can get to the Z3-str family and utilizes a word equation reduction approach to reason about regular expressions. Z3str3RE\cite{z3str3re} is a variant of Z3str3 that incorporates length-aware algorithms and heuristics. Z3seq\cite{z3seq} is a sequence solver which uses a novel derivative theory for solving extended regular expressions. Z3-Trau\cite{z3trau} is the Z3 version of trau\cite{trau} that employs a flat automata-based approach, incorporating both under- and over-approximations. Ostrich\cite{ostrich} is the tool we extend which uses automaton to model the semantics of string functions and regular memberships. We used the 1.0.5 binary version of CVC5, commit 59e9c87 of Z3str3, last version of Z3str3RE, 4.8.9 binary version of Z3Seq, commit 1628747 of Z3-Trau and commit 8297d8d of Ostrich. Z3-Trau does not support \verb|re.diff|. All other solvers support all syntax sugars listed in SMT-LIB standard\cite{smt_lib}. We omitted Z3str4\cite{z3str4} because the provided reproduction package link is wrong. All experiments are conducted on CentOS Stream release 8 with 12 Intel(R) Xeon(R) Platinum 8269CY CPU T 3.10GHz processors and 190 GB memory. We used Zaligvinder\cite{zaligvinder_2021} framework and set the timeout to 60 seconds. 

%\subsection{Overall Evaluation}
In Fig.\ref{fig:cactus_all}, the cactus plot illustrates the cumulative time each solver takes for all cases in ascending order of runtime. Solvers located towards the right and lower portion of the plot indicate better performance. \newline
Table \ref{tab:results_all} summarizes the results that demonstrate OstrichCEA's superior performance, solving the most significant number of instances and outperforming most competing solvers. Including timeouts, OstrichCEA is \textbf{1.52}\mult{} faster than Z3Seq, \textbf{2.23}\mult{} faster than Ostrich, \textbf{2.26}\mult{} faster than Z3-Trau, \textbf{3.08}\mult{} faster than Z3str3, \textbf{3.11}\mult{} faster than CVC5 and close to Z3str3RE with \%2 speed loss . Note that CVC5\cite{cvc5} yielded 5370 timeouts(11\% of all instances), and Z3str3\cite{z3str3} yielded 6139 timeouts(12\% of all instances), which is much more than other solvers. Both CVC5 and Z3str3 are DPLL(T)-based solvers. It seems that almost 10\% of the benchmarks we used seem unsuitable for them. Z3-trau\cite{z3trau} yielded 21,152 unknowns because it does not support \verb|re.diff|. Z3-trau\cite{z3trau} yielded 6673 crashes (13\% of the instances) and 1233 soundness errors (2\% of the instances), which is a large portion. Z3-based solvers Z3str3 yielded 38 soundness errors, Z3seq\cite{z3seq} yielded 51 soundness errors and Z3str3RE\cite{z3str3re} yielded 39 soundness error. Most of them are due to the mistake formalization of the backslash character. OstrichCEA resulted 5 unknowns because it reached the maximum threshold of limited memory 2GB. More details of the results of each benchmark are shown in \ref{appendix:experiential_results}.
\begin{figure}[h]
  \centering
  \import{figures}{cactus_plot_all.tex}
  \caption{Cactus plot summarizing performance on all benchmarks.}
  \label{fig:cactus_all}
\end{figure}

%\subsection{Analysis of Individual Heuristics}
In order to demonstrate the efficacy of the individual heuristics outlined in Section \ref{sec:algorithm} and incorporated into OstrichCEA, we assessed various tool configurations in which one or more heuristics were disabled. Figure \ref{fig:cactus_heuristics} and Table \ref{tab:results_heuristics} display the outcomes. The "OstrichCEA" plot line represents the tool's performance when all heuristics are enabled, while the "All heuristics off" line represents performance when all heuristics are disabled. The remaining plot lines exhibit performance with only the named heuristic disabled while all others are enabled. From the plots and table, it is evident that OstrichCEA functions most effectively when all heuristics are enabled. In average time with timeout times, OstrichCEA performs \textbf{1.78}$\times$ faster than employing none of our heuristics, \textbf{1.37}$\times$ faster than turning off simplification, \textbf{1.15}$\times$ faster than truing off under-approximation.
\begin{figure}
  \subimport{figures/}{cactus_plot_heuristic.tex}
  \caption{A performance comparison was made on all benchmarks by disabling individual heuristics using a cactus plot.}
  \label{fig:cactus_heuristics}
\end{figure}
\begin{table}
  \subimport{tables}{table_heuristic.tex}
  \caption{A performance comparison was made on all benchmarks by disabling individual heuristics.}
  \label{tab:results_heuristics}
\end{table}
}
%%%%%%%%%% original texts by denghang %%%%%%%%%
%%%%%%%%%% original texts by denghang %%%%%%%%%
%\end{document}