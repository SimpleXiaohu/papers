\documentclass{standalone}
\begin{document}
This section presents the empirical evaluation of OstrichCEA, which is our implementation of the decision procedure introduced in Section \ref{sec:algorithm}. Our objective is to validate the effectiveness of the proposed techniques by evaluating our tool's correctness and efficiency compared to other solvers. Furthermore, we assess the efficacy of our heuristics by testing different configurations of the tool. We have implemented our encoding for bounded repetition with two heuristic algorithms on Ostrich+ \cite{atva2020}. The pre-image computation for concatenation, \verb|indexOf|, \verb|substring|, \verb|replaceAll|, \verb|reverse| and finite transducer remain unchanged. OstrichCEA is written in Scala and based on the SMT solver Princess\cite{princess}.
\subsection{Benchmarks}
We conducted a comparison on four sets of benchmarks based on regex with bounded repetition, consisting of a total of 49,379 instances. We analyze all 19 developed benchmarks listed in \cite{zaligvinder_2021} and find that only \textbf{AutomatArk} benchmarks have regular membership with bounded repetition. In total, almost 18\% of the instances we evaluated were sourced from published industrial benchmarks or other solver developers. All the other instances contain regular expressions from the real world. Each set of the benchmark is evenly divided into "large" and "small": the "large" set contains instances with large repetition upper bounds (the sum of upper bounds is greater than 50), and "small" contains the other instances. Less than 10\% of the benchmarks are in a "large" set. More details about the benchmarks are shown below.
\subsubsection{AutomatArk} is the 8,751 instances generated by Berzish et al.\cite{z3str3re}. It is based on real-world regular expression queries from Loris D'Antoni\cite{automatark}. The origin set comprises two tracks, a simple and a hard track, with 19,979 instances. The simple track contains instances with a single regular expression membership constraint, whereas the hard track can hold up to five membership constraints for a single variable per instance. We extract 8,751 instances containing bounded repetition from 19,979 instances and partition them into "large" and "small" in terms of the size of the bounds.
\subsubsection{ReDos} is the set of 1,624 instances we generated. It is based on the ReDos-attacked regular expression collected by Lenka et al. For each regular expression, we generate an instance as the template (\ref{eq:template}) where $\regex$ is the regular expression. The regular membership predicate $x\not\in \Sigma^*(<\mid >\mid '\mid ''\mid \&)\Sigma^*$ sanitizes the input string $x$ to avoid the attack. The length lower bound is set to $20$ for the "small" set and $50$ for the "large" set.
\subsubsection{RegexLib} is the set of 1,623 instances we generated similarly to the \textbf{ReDos} benchmark. It is based on the regular expressions collected by James C. Davis et al.\cite{regex_lingua_franca} from regex lib website\cite{regexlib}. The website is the Internet's first Regular Expression Library. Currently, it has indexed 4149 expressions from 2818 contributors around the world since 2001. We extract 1,623 instances containing bounded repetition from 4149 instances and partition them into "large" and "small" in terms of the size of the bounds.
\subsubsection{StackOverflow} is the 37381 instances we generated similarly to the \textbf{ReDos} benchmark. As the Regexlib benchmark, the real-world regex expressions are collected by James C. Davis et al.\cite{regex_lingua_franca} from StackOverflow website\cite{stackoverflow}. The website is a question-and-answer site for professional and enthusiast programmers. We extract 37,381 instances containing bounded repetition from almost 500,000 instances and partition them into "large" and "small" in terms of the size of the bounds.
\begin{equation} \label{eq:template}
  x\in \regex \wedge x\not\in \Sigma^*(<\mid >\mid '\mid ''\mid \&)\Sigma^*\wedge |x| > 50(20)
\end{equation}
\subsection{Experimental Setup and Compared Solvers}
We have evaluated OstrichCEA compared to five other prominent string solvers currently available. We evaluate the solvers by directly comparing the number of cases correctly solved, the total time taken with and without timeouts, and the total count of soundness errors and program crashes. One of these solvers is CVC5\cite{cvc5}, a general SMT solver that uses algebraic reasoning to handle strings and regular expressions and is the winner of SMT-COMP 2022\cite{smt-comp}. Another solver, Z3str3\cite{z3str3}, is the most recent addition to the Z3-str family and utilizes a word equation reduction approach to reason about regular expressions. Z3str3RE\cite{z3str3re} is a variant of Z3str3 that incorporates length-aware algorithms and heuristics. Z3seq\cite{z3seq} is a sequence solver which uses a novel derivative theory for solving extended regular expressions. Z3-Trau\cite{z3trau} is the Z3 version of trau\cite{trau} that employs a flat automata-based approach, incorporating both under- and over-approximations. Finally, Ostrich\cite{ostrich} is the tool we extend which uses automaton to model the semantics of string functions and regular memberships. We used the 1.0.5 binary version of CVC5, commit 59e9c87 of Z3str3, last version of Z3str3RE, 4.8.9 binary version of Z3Seq, commit 1628747 of Z3-Trau and commit 8297d8d of Ostrich. Z3-Trau does not support \verb|re.diff|. All other solvers support all syntax sugars listed in SMT-LIB standard\cite{smt_lib}. We omitted Z3str4\cite{z3str4} because the provided reproduction package link is wrong. All experiments are conducted on CentOS Stream release 8 with 12 Intel(R) Xeon(R) Platinum 8269CY CPU T 3.10GHz processors and 190 GB memory. We used Zaligvinder\cite{zaligvinder_2021} framework and set the timeout to 60 seconds. 
\subsection{Overall Evaluation}
In Fig.\ref{fig:cactus_all}, the cactus plot illustrates the cumulative time each solver takes for all cases in ascending order of runtime. Solvers located towards the right and lower portion of the plot indicate better performance. OstrichCEA spent more time solving instances than other solvers. One reason was that OstrichCEA was written in Scala and ran on a Java virtual machine(JVM). The cold boot of JVM is cost. The other reason was that the implemented decision procedure was excessively cumbersome in resolving certain straightforward constraints. \newline
Table \ref{tab:results_all} summarizes the results that demonstrate OstrichCEA's superior performance, solving the most significant number of instances and outperforming most competing solvers. Including timeouts, OstrichCEA is \textbf{1.52}\mult{} faster than Z3Seq, \textbf{2.23}\mult{} faster than Ostrich, \textbf{2.26}\mult{} faster than Z3-Trau, \textbf{3.08}\mult{} faster than Z3str3, \textbf{3.11}\mult{} faster than CVC5 and close to Z3str3RE with \%2 speed loss . Note that CVC5\cite{cvc5} yielded 5370 timeouts(11\% of all instances), and Z3str3\cite{z3str3} yielded 6139 timeouts(12\% of all instances), which is much more than other solvers. Both CVC5 and Z3str3 are DPLL(T)-based solvers. It seems that almost 10\% of the benchmarks we used seem unsuitable for them. Z3-trau\cite{z3trau} yielded 21,152 unknowns because it does not support \verb|re.diff|. Z3-trau\cite{z3trau} yielded 6673 crashes (13\% of the instances) and 1233 soundness errors (2\% of the instances), which is a large portion. Z3-based solvers Z3str3 yielded 38 soundness errors, Z3seq\cite{z3seq} yielded 51 soundness errors and Z3str3RE\cite{z3str3re} yielded 39 soundness error. Most of them are due to the mistake formalization of the backslash character. OstrichCEA resulted 5 unknowns because it reached the maximum threshold of limited memory 2GB.

\begin{figure}[h]
  \centering
  \import{figures}{cactus_plot_all.tex}
  \caption{Cactus plot summarizing performance on all benchmarks.}
  \label{fig:cactus_all}
\end{figure}
\begin{table}
  \import{tables}{table_all.tex}
  \caption{Total results of string solvers on all benchmarks. OstrichCEA solved the most benchmarks in the second shortest time.}
  \label{tab:results_all}
\end{table}

\subsection{Detailed Evaluation}
The detailed results for the \textbf{AutomatArk} benchmark are presented in Figure \ref{fig:cactus_automatrk} and Table \ref{tab:results_automatrk}. OstrichCEA solved the most \emph{unsat} instances. Z3str3RE solved the most \emph{sat} instances. Ostrich solved the greatest number of instances, while Z3str3RE used the least time with timeouts.\newline
The detailed results for the \textbf{Redos} benchmark are presented in Figure \ref{fig:cactus_redos} and Table \ref{tab:results_redos}. OstrichCEA solved the most instances on both tracks of \emph{sat} and \emph{unsat}. Including timeouts, OstrichCEA is \textbf{4.4}\mult{} faster than CVC5, \textbf{3.74}\mult{} faster than Ostrich, \textbf{2.09}\mult{} faster than Z3str3, \textbf{5.40}\mult{} faster than Z3seq, \textbf{2.49}\mult{} faster than Z3-Trau and close to Z3str3RE with \%20 speed loss. \newline 
The detailed results for the \textbf{RegexLib} benchmark are presented in Figure \ref{fig:cactus_regexlib} and Table \ref{tab:results_regexlib}. Ostrich solved the most \emph{unsat} instances. OstrichCEA solved the most \emph{sat} instances. In total, OstrichCEA solved the greatest number of instances and was the solver with medium speed. \newline
The detailed results for the \textbf{StackOverflow} benchmark are presented in Figure \ref{fig:cactus_stackoverflow} and Table \ref{tab:results_stackoverflow}. OstrichCEA solved the most instances on both tracks of \emph{sat} and \emph{unsat} and was the faster solver.\newline Including timeouts, OstrichCEA is \textbf{3.98}\mult{} faster than CVC5, \textbf{2.63}\mult{} faster than Ostrich, \textbf{2.19}\mult{} faster than Z3str3, \textbf{1.59}\mult{} faster than Z3seq, \textbf{1.28}\mult{} faster than CVC5 and \textbf{1.47}\mult{} to Z3str3RE. \newline
\begin{figure}
  \centering
  \subimport{figures}{cactus_plot_automatark.tex}
  \caption{The plot of a cactus graph depicting a comprehensive evaluation of the performance of the AutomatArk benchmark.}
  \label{fig:cactus_automatrk}
\end{figure}
\begin{table}
  \subimport{tables}{table_automatark.tex}
  \caption{Detailed results for the AutomatArk benchmark.}
  \label{tab:results_automatrk}
\end{table}

\begin{figure}
  \subimport{figures}{cactus_plot_redos.tex}
  \caption{The plot of a cactus graph depicting a comprehensive evaluation of the performance of the ReDos benchmark.}
  \label{fig:cactus_redos}
\end{figure}
\begin{table}
  \subimport{tables}{table_redos.tex}
  \caption{Detailed results for the ReDos benchmark.}
  \label{tab:results_redos}
\end{table}

\begin{figure}
  \subimport{figures}{cactus_plot_regexlib.tex}
  \caption{The plot of a cactus graph depicting a comprehensive evaluation of the performance of the RegexLib benchmark.}
  \label{fig:cactus_regexlib}
\end{figure}
\begin{table}
  \subimport{tables}{table_regexlib.tex}
  \caption{Detailed results for the RegexLib benchmark.}
  \label{tab:results_regexlib}
\end{table}

\begin{figure}
  \subimport{figures/}{cactus_plot_stackoverflow.tex}
  \caption{The plot of a cactus graph depicting a comprehensive evaluation of the performance of the StackOverflow benchmark.}
  \label{fig:cactus_stackoverflow}
\end{figure}
\begin{table}
  \subimport{tables}{table_stackoverflow.tex}
  \caption{Detailed results for the StackOverflow benchmark.}
  \label{tab:results_stackoverflow}
\end{table}
\subsection{Analysis of Individual Heuristics}
In order to demonstrate the efficacy of the individual heuristics outlined in Section \ref{sec:algorithm} and incorporated into OstrichCEA, we assessed various tool configurations in which one or more heuristics were disabled. Figure \ref{fig:cactus_heuristics} and Table \ref{tab:results_heuristics} display the outcomes. The "OstrichCEA" plot line represents the tool's performance when all heuristics are enabled, while the "All heuristics off" line represents performance when all heuristics are disabled. The remaining plot lines exhibit performance with only the named heuristic disabled while all others are enabled. From the plots and table, it is evident that OstrichCEA functions most effectively when all heuristics are enabled. OstrichCEA performs 1.78 times faster when employing all of our heuristics than none. All other configurations of the tool's heuristics make sense in comparison.
\begin{figure}
  \subimport{figures/}{cactus_plot_heuristic.tex}
  \caption{A performance comparison was made on all benchmarks by disabling individual heuristics using a cactus plot.}
  \label{fig:cactus_heuristics}
\end{figure}
\begin{table}
  \subimport{tables}{table_heuristic.tex}
  \caption{A performance comparison was made on all benchmarks by disabling individual heuristics.}
  \label{tab:results_heuristics}
\end{table}

\end{document}